\chapter{OVERVIEW ABOUT PROJECT}

\renewcommand{\headrulewidth}{0.5pt}
\renewcommand{\footrulewidth}{0.5pt}
\thispagestyle{plain}
\pagestyle{fancy}
\fancyhf{}
\fancyhead[L]{\textbf{CHAPTER 1}}
\fancyhead[R]{\textbf{DROWSINESS DETECTION AND ALERT SYSTEM IN THE CAR}}
\raggedright
\fancyfoot[L]{From: Nguyen Van Anh Tuan}
\fancyfoot[R]{Page \thepage}

\justifying

\section{Introduction}
    Nowaday along with the strong development of Science Technology, Robot, Self-Driving Car, 
    AI,... In addition, image processing is a relatively new science compared to many other sciences, but now 
    it is one of the rapidly growing fields and attracts special attention from researchers, research centers, 
    application on this fascinating field. Image processing plays an important role in many practical applications 
    of science and technology as well as in everyday life such as: production and quality assurance, movement of robot, 
    self-driving car, guild tool for the blind, security and monitoring,... \\ 
    \vspace{3mm}
    Recently, the popularity and efficiency of using Raspberry Pi kit in applications in science and technology, 
    with characteristics like a miniature computer about the size of a mobile phone, runs an open operating system, 
    is equiped with a powerful processor, low power consumption, and low cost, allowing you to configure the Raspberry 
    Pi kit as a problem-solving computer. \\ 
    \vspace{3mm}
    Besides, from the actual needs, drowsiness while driving is quite common and it is also one of the casues of serious 
    accidents, requiring a device that can monitor the state of the person while driving to be able to promptly 
    warn the driver when the driver accidentally falls asleep while driving. \\ 
    \vspace{3mm}
    From these reasons has prompted me to research application of Raspberry Pi kit to image processing in order to offer 
    some image processing solutions that can be applied in life.

\section{Target and The Limits of Project}
    This project is the first step to learn about the application of processed images in reality, at the same time is also 
    a step to deploy the learned knowledge. Through research and serious work to practice manners, as well as perfecting 
    methods, researching thinking and solving a problem. With the objectives of the project is:
    \begin{itemize}
        \item Learning about Raspberry Pi 3 model B+ kit
        \item Install OS for Raspberry Pi 3 B+
        \item Learn about image processing
        \item Learn about OpenCV, Python
        \item Install library for OpenCV, Dlib
        \item Recognize techniques
        \item Drowsiness Detection by using camera connect to Raspberry Pi and alert to driver through speaker
        \item Write program
        \item Experimental model
        \item Write report
    \end{itemize}
    The limit of project is the distance from camera to object from 0,3-1m, detected object not to use glasses and 
    the angle is smaller than 40 degrees, if the object is out of this range, the detection maybe inaccurate or undetectable.